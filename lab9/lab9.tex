\documentclass[a4paper]{article}

%% Language and font encodings
\usepackage[english]{babel}
\usepackage[utf8x]{inputenc}
\usepackage[T1]{fontenc}

%% Sets page size and margins
\usepackage[a4paper,top=3cm,bottom=2cm,left=3cm,right=3cm,marginparwidth=1.75cm]{geometry}

%% Useful packages
\usepackage{amsmath}
\usepackage{graphicx}
\usepackage[colorinlistoftodos]{todonotes}
\usepackage[colorlinks=true, allcolors=blue]{hyperref}

\title{Simple Linear Regression Report}
\author{Cheng Peng}

\usepackage{Sweave}
\begin{document}
\Sconcordance{concordance:lab9.tex:lab9.Rnw:%
1 19 1 1 0 56 1}

\maketitle

\begin{abstract}
This report aims to reproduce the main results displayed in section 3.1: Simple Linear Regression of the book An Introduction to Statistical Learning.
\end{abstract}

\section{Introduction}
According to the book, the overall goal is to provide advice on how to improve sales of the particular product. More specifically, the idea is to determine whether there is an association between advertising and sales, and if so, develop an accurate model that can be used to predict sales on the basis of the three media budgets. We therefore fit a simple linear regression model, as discussed in the methodology part to analyze such association.

\section{Data}
The data set consists of the Sales (in thousands of units) of a particular product in 200 different markets, along with advertising budgets (in thousands of dollars) for the product in each of those markets for three different media: TV, Radio, and Newspaper.


\section{Methodology}
In this paper, we mainly consider the relatinoship between Sales and one media from the data set, TV. In order to explore this relationship, we use a simple linear model and regress sales onto TV by fitting the model:

\begin{equation}
Sales = \beta_0 + \beta_1 TV
\end{equation}

With this linear model, we estimate the coefficients by minimizing the least squares criterion, which is minimizing the sum of squared errors.

\section{Results}

With the least square estimators, we compute the regression coefficients.\newline
Table 1: Information about Regression Coefficients

\vspace{2mm}
\begin{tabular}{ | p{3cm} | p{2cm} | p{2cm} | p{2cm} | p{2cm} |}
  \hline			
  Coefficients & Estimate & Std. Error & t-statistics & Pr Value \\
  Intercept & 7.0325 & 0.4578 & 15.36 & <0.00 \\
  TV & 0.0475 & 0.0027 & 17.67 & <0.00 \\
  \hline  
\end{tabular}

\vspace{5mm}

More information about the least squares model is given in the table below: \newline
Table 2: Regression Quality Indices

\vspace{2mm}
\begin{tabular}{ | p{4cm} | p{2cm} | }
  \hline			
  Quantity & Value \\
  Residual Standard Error & 3.259 \\
  R-squared & 0.612 \\
  F-statistic & 312.14 \\
  \hline  
\end{tabular}

\section{Conclusion}
That's it.

\end{document}
